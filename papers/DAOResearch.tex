\documentclass[12pt,a4paper]{article}
\usepackage[utf8]{inputenc}
\usepackage{amsmath}
\usepackage{amsfonts}
\usepackage{amssymb}
\usepackage{graphicx}
\usepackage{hyperref}
\usepackage{xcolor}
\usepackage[left=2cm,right=2cm,top=2cm,bottom=2cm]{geometry}

% Define custom command for text reference
\newcommand{\textref}[2]{\hyperref[#1]{#2}}

\definecolor{lightgray}{rgb}{0.9, 0.9, 0.9}

\setlength{\parindent}{0pt}

\begin{document}

\begin{titlepage}
    \centering
    
    \vspace*{10cm}

    \Large
    \textbf{DAO Research} \\ 
    
    \vfill
    
    2023
    
\end{titlepage}

\begin{abstract}
It is my summary of what I have learned about DAO. 
\end{abstract}

\section{Introduction}

\subsection{Definitions}

Decentralized autonomous organizations (DAOs) first began to surface in discussions of the blockchain community around 2013, where people imagined them as digital substitutes for traditional organizations.

\vspace*{0.5cm}

\textbf{The key arguments:}
\begin{itemize}
    \item Automation of many organization processes
    \item Allow for more broad-based ownership and governance of the digital economy
\end{itemize}
 
They are based on a cryptographically secured blockchain.

\vspace*{0.5cm}

Driven by the surge of interest in crypto, real-world deployment of DAOs surged by as much as 660\% from 2019 to late 2020. \\

While there is no single comprehensive definition of DAOs, their core characteristics can be synthesized from various academic uses of the term.
\begin{itemize}
    \item DAOs are organizations governed by a smart contract, typically deployed on a blockchain that autonomously enforces rules for interaction among the members.
    \item (DAOs also belong to a larger class of digitally-constituted organizations) organizations governed through computational artifacts such as software, hardware, and/or protocols.
\end{itemize}

For example, the Bitcoin, Ethereum, and Tor networks, even though they are not DAOs, \colorbox{lightgray}{can be classified as digitally constituted organizations}.

\subsection{The bigger picture}

As an organizational form, DAOs are closely related to earlier forms of online community, especially open-source communities; many of the most successful DAOs are also communities organized around an open-source product. 

Antecedents in digital cooperatives of DAOs: keiretsus, Patreon, World of Warcraft and Second Life, Wikipedia.

\vspace{0.5cm}

The uniqueness of DAO:
\begin{itemize}
    \item Intervention afforded by smart contract (enforcing rules)
    \item The native availability of fine-grained data (existing protocols?)
    \item The vibrancy of the ecosystem (liveness property) 
    \item The fast-moving do-ocratic culture inherited from tech and open-source (move fast and break things)
\end{itemize}

The distinguishing features of DAO:
\begin{itemize}
    \item More financially sophisticated (DeFi, Tokens, etc.)
    \item More entrepreneurial
    \item Less permissioned by existing institutions
\end{itemize}

Comparison with different fields
\begin{itemize}
    \item In comparison to the blockchain (with which it substantially overlaps), DAO science is more experimental and human-centred.
    \item In comparison to older studies of online communities and virtual (game) worlds, DAO science requires significantly more legal, economic, and organizational sophistication; in DAOs, the scope of activities is broader, and the stakes are higher. 
    \item In comparison to the many studies of information technology within management science, DAO science is more keyed into questions of governance rather than operations (DAO or digitally-constituted organization may govern incentives and roles within a co-op).
\end{itemize}

Therefore, DAOs can express the full range of existing organizational forms—so the choice is not between a corporation and a DAO per se but between a traditional, legally constituted corporation and a corporation that is digitally constituted through a smart contract. 

\newpage

\section{Usages}

\end{document}